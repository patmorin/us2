\documentclass{patmorin}
\listfiles


\definecolor{brightmaroon}{rgb}{0.76, 0.13, 0.28}
\definecolor{linkblue}{rgb}{0, 0.337, 0.227}
\newcommand{\defin}[1]{\emph{\color{brightmaroon}#1}}
\setlength{\parskip}{1ex}

\DeclareMathOperator{\outn}{out}
\DeclareMathOperator{\comp}{comp}

\title{\MakeUppercase{A Unique Superior is Still One Too Many}}
\author{Bra\v{c} 2023 Gang}


\newcommand{\rn}[1]{\chi_{\operatorname{#1-vr}}}
\newcommand{\irn}{\rn{\infty}}
% \newcommand{\trn}{\rn{2}}
\newcommand{\trn}{\chi_{\mathrm{us}}}
\newcommand{\lrn}{\rn{\ell}}
\newcommand{\dtcn}{\bar{\chi}_2}
\newcommand{\dlcn}{\bar{\chi}_\ell}
\newcommand{\scn}{\chi_{\star}}


% \pagenumbering{roman}
\begin{document}

\maketitle

\begin{abstract}
  A \defin{unique-superior colouring} (also known as a \defin{$2$-vertex-ranking} or \defin{restricted star colouring}) is a mapping $\varphi:V(G)\to\N$ of the vertices of a graph $G$ to integer colours so that for any edge $uv$, $\varphi(u)\neq \varphi(v)$ and for any $3$-vertex path $uvw$, $\phi(u)\neq\varphi(w)$ or $\varphi(v)>\varphi(u)$.  For a graph $G$, the unique-superior chromatic number $\trn(G)$ is the minimum value of $k$ such that $G$ has a us-colouring $\varphi:V(G)\to\{1,\ldots,k\}$.  We show that there exists a constant $c$ such that every $n$-vertex $d$-degenerate graph $G$ has $\trn(G) \le cd^{2/3} n^{1/3}\log n$.  These results extend to the $\ell$-vertex-ranking problem in which one considers paths $v_0,\ldots,v_r$ of length up to $\ell$ and requires that $\varphi(v_0)\neq\varphi(v_r)$ or that $\varphi(v_0)<\max\{\varphi(v_1),\ldots,\varphi(v_{r-1})\}$.  In this setting, for any fixed $\ell$ and $d$, the bound on the $\ell$-vertex-ranking number of $d$-degenerate graphs is $\lrn(G)\in O(n^{(\ell-1)/(\ell+1)}\log n)$.
\end{abstract}


% \tableofcontents
%
% \newpage
% \pagenumbering{arabic}



\section{Introduction}

A \defin{unique-superior colouring} (also known as a \defin{$2$-vertex-ranking} or \defin{restricted star colouring}) is a mapping $\varphi:V(G)\to\N$ of the vertices of a graph $G$ to integer colours so that for any edge $uv$, $\varphi(u)\neq \varphi(v)$ and for any $3$-vertex path $uvw$, $\varphi(u)\neq\varphi(w)$ or $\varphi(v)>\varphi(u)$.  For a graph $G$, the unique-superior chromatic number $\trn(G)$ is the minimum value of $k$ such that $G$ has a us-colouring $\varphi:V(G)\to\{1,\ldots,k\}$.

\citet{karpas.neiman.ea:on} showed that for every $d$-degenerate $n$-vertex graph $G$, $\trn(G)\in O(d\sqrt{n})$ and that there exists $2$-degenerate $n$-vertex graphs with $\trn(G)\in\Omega(n^{1/3})$.  They leave the question of closing the gap between these bounds as an open problem. In the current paper we give a colouring procedure that essentially matches their lower bound to within a polylogarithmic factor:

\begin{thm}\label{d_degenerate_upper_bound}
  There exists a constant $c>0$ such that for any integer $d\ge 1$, every $n$-vertex $d$-degenerate graph $G$ has $\trn(G) \le c d^{2/3} n^{1/3}\log  n$.
\end{thm}

Like the upper bound in \cite{karpas.neiman.ea:on} \cref{d_degenerate_upper_bound} follows quickly for a theorem about graphs that are both $d$-degenerate and have maximum-degree $\Delta$:

\begin{thm}\label{degenerate_and_degree}
  There exists a fixed $c>0$ such that
  for all integers $d,\Delta \ge 1$, and $n> \Delta$, every $d$-degenerate $n$-vertex graph $G$ of maximum degree at most $\Delta$ has
  $\trn(G)\leq c \sqrt{d\Delta}\log^{3/2}  n$.
\end{thm}

\Cref{d_degenerate_upper_bound} follows easily from \cref{degenerate_and_degree}, by the following argument:  Since $G$ is $d$-degenerate, it has at most $dn$ edges and the sum of vertex degrees is at most $2dn$.  Let $\Delta:=d^{1/3}n^{2/3}/\log n$.  Then the set $S:=\{v\in V(G):\deg_G(v)\ge \Delta\}$ has size at most $2dn/\Delta=2d^{2/3}n^{1/3}\log n$.  Since $G-S$ has maximum degree $\Delta$, $G-S$ has a us-colouring $\varphi$ using at most $c\sqrt{d\Delta}\log^{3/2} n = cd^{2/3}n^{1/3}\log  n$ colours, by \cref{degenerate_and_degree}. We can extend $\varphi$ to a colouring of $G$ by assigning each vertex in $S$ a distinct colour that is larger than any colour used in the colouring of $G-S$.  Thus, $\trn(G)\le cd^{2/3}n^{1/3}\log  n+2d^{2/3}n^{1/3}= (c+2)d^{2/3}n^{1/3}\log  n$.  In the next section, we prove \cref{degenerate_and_degree}.

\todo[inline]{I think if we're a little more careful we can get a better dependence on $d$.  I don't care much, except that $(dn)^{1/3}\log n$ is much prettier than $d^{2/3}n^{1/3}\log n$.}

\section{The Proof}

We use standard graph-theoretic terminology and conventions as used by \citet{diestel:graph}. A graph $G$ has vertex set $V(G)$ and edge set $E(G)$.  For an integer $\ell$, $G^\ell$ denotes the graph with vertex set $V(G)$ that contains an edge $vw$ if and only if $G$ contains a path of length at most $\ell$ with endpoints $v$ and $w$.

\begin{proof}[Proof of \cref{degenerate_and_degree}]
  Let $S_0:=V(G)$ and, for each integer $i\ge 1$, let $S_i:=\{v\in S_{i-1}:\deg_{G[S_{i-1}]}(v)\ge 4d\}$.  Since $G$ is $d$-degenerate $G[S_{i-1}]$ has at most $d|S_{i-1}|$ edges and the sum of all vertex degrees in $G[S_{i-1}]$is at most $2d|S_{i-1}|$, so $|S_i|\le |S_{i-1}|/2$ for each $i\ge 1$.  Let $k$ be the maximum integer such that $S_i$ is non-empty.  For each $i\in\{0,\ldots,k\}$, let $L_i:=S_i\setminus S_{i+1}$.  (These notations are mnemonics: $S_i$ are the \defin{survivors} of round $i-1$ and $L_i$ is \defin{layer} $i$.)

  % We will construct a us-colouring $\varphi$ of $G$ using a product colouring.  For each $i\in\{0,\ldots,k\}$, we compute a proper colouring of the graph $(G[L_i])^2$.  Since each vertex in $G[L_i]$ has degree less than $4d$,
  % % in $G[S_i]\subseteq G[L_i]$,
  % this colouring requires at most $16d^2$ colours.  This colouring is one factor in the product colouring of $G$.
  % \todo[inline]{During the next round of revisions, we can get rid of this second colouring.  An appropriate definition of \defin{problematic} vertices will capture both monochromatic edges and three-vertex paths within a single layer.}

  We compute our colouring using a two-phase process. In the first phase we use a sequence of pairwise-disjoint color palettes $\Phi_0,\ldots,\Phi_{k}$, each of size $2\sqrt{\Delta}$, such that for each $1\le i < j\le k$, every colour in $\Phi_i$ is less than every colour in $\Phi_j$.  We will use the colours in $\Phi_i$ to colour the vertices in $L_i$, for each $i\in\{0,\ldots,k\}$.

  % where $\Phi$
  % where $\Phi_i:=\{2i\sqrt{\Delta}+1,\ldots,2(i+1)\sqrt{\Delta}\}$ and is used to colour the vertices of $L_i$, for each $i\in\{0,\ldots,k\}$.

  Fix an arbitrary total order $<$ on the vertices of $S_i$ for each $i\in\{0,\ldots,k\}$ and define the total order $<$ on $V(G)$ in which $v <w$ if $v\in L_i$, $w\in L_j$, and $i<j$ or if $v,w\in L_i$ and $v<w$.  Let $H$ be the directed acyclic graph obtained from $G^2$ by directing each edge $vw$ as $\overrightarrow{vw}$ so that $v<w$.  We claim that each vertex $v$ has out-degree at most $(4d-1)(2\Delta+1)$ in $H$.  To see this, consider some edge $\overrightarrow{vw}$ and suppose $v\in L_i$.  The existence of $\overrightarrow{vw}$ in $H$ implies at least one of the following:
  \begin{compactenum}
    \item $vw\in E(G)$.  Since $v<w$ and $v\in L_i$, $w\in S_i$.  Since $v\in L_i$, $\deg_{G[S_i]}(v)\le 4d-1$, so this type of edge contributes at most $4d-1$ to the out-degree of $v$.
    \item $G$ contains a path $vyw$ with $y < v < w$ or $v < y < w$.  Since $\deg_G(v)\le\Delta$, there are at most $\Delta$ choices for $y$.  For each such $y$ there are at most $4d-1$ choices for $w$.  Therefore, these types of vertices contribute at most $(4d-1)\Delta$ to the out-degree of $v$.
    \item $G$ contains a path $vyw$ with $v < w < y$.  Since $v<y$ there are at most $4d-1$ choices for $y$ and for each such $y$, $\deg_G(y)\le\Delta$, so there are at most $\Delta$ choices for $w$.  Therefore, these types of edges contribute at most $(4d-1)\Delta$ to the out-degree of $v$.
  \end{compactenum}
  We colour the vertices of $G$ in the order given by $<$.  When colouring some vertex $w\in L_i$ we count, for each colour $\alpha\in \Phi_i$, the number $N_\alpha(w)$ of neighbours $y\in N_G(w)$ such that $y < w$ and $y$ has colour $\alpha$ or $y$ already has a neighbour $u$ of colour $\alpha$.  We choose the colour of $w$ uniformly at random from a subpalette of $\Phi_i$ that contains at least half the colours in $\Phi_i$.  Specifically we choose the colour of $w$ from the subpalette $\Phi_w:=\{\alpha\in\Phi_i: N_{\alpha}(w)<\sqrt{\Delta}\}$.  This subpalette contains at least $\sqrt{\Delta}$ colours because the number of $\alpha\in\Phi_i$ with $N_\alpha(w)\ge \sqrt{\Delta}$ is at most $\sqrt{\Delta}$ (since $|N_G(w)|\leq \Delta$).
  % Since $\sum_{\alpha\in\Phi_i} N_\alpha(w)\le\deg_G(w)\le\Delta$, the set $\Phi_w$ has size at least $2\sqrt{\Delta}-\sqrt{\Delta}=\sqrt{\Delta}$.
  This completes the description of the first-phase colouring $\varphi$ of $G$.

  % The resulting product colouring is a proper colouring because each edge $vw$ of $G$ either has both endpoints in $L_i$ for some $i\in\{0,\ldots,k\}$, in which case $v$ and $w$ have different colours in the first factor of the product colouring or the second factor in the colours of $v$ and $w$ are taken from disjoint palettes.  We claim that the only violations of the us-colouring conditions occur at three-vertex paths $vyw$ with $v,w\in L_j$ for some $j\in\{0,\ldots,k\}$ and $y\in L_i$ for some $j\in\{0,\ldots,j-1\}$. Indeed, since the colour of $v$ and $w$ are the same, $v$ and $w$ must be in the same layer $L_j$ and $y\not\in L_j$.  Since the colour of $v$ and $w$ is larger than that of $y$, $j> i$.

  We say that a vertex $y$ is \defin{problematic} if $y$ has a neighbour of the same colour or if $y$ has neighbours $u$ and $w$ in $G$ such that $y\in L_i$, $u,w\in L_j$, $i \le j$ and $u$ and $w$ receive the same colour in phase one.  At this point, we can offer some justification for the definition of the colour set $\Phi_w$ that we choose from when colouring $w$. For each problematic vertex $y$, there exists a minimum vertex $w$ (with respect to $<$) such that $y$ becomes problematic precisely when $w$ is coloured.  When this happens, we say that $w$ \defin{completes} $y$. By definition, if $w$ chooses the colour $\alpha\in\Phi_w$, then it completes at most $|N_{\alpha}(w)|<\sqrt{\Delta}$ vertices.



  Let $P$ be the set of all problematic vertices in $G$.  We will re-colour every vertex in $P$ with a colour in a palette $\Phi_{k+1}$ of size $Cd\sqrt{\Delta}\log n+1$ whose colours are all larger than all colours in $\Phi_0,\ldots,\Phi_k$.  Since we are recolouring vertices in $P$ with large colours in $\Phi_{k+1}$, the only violations that could occur after recolouring would be caused by paths (with two or three vertices) whose endpoints are in $P$.  In order to avoid these we will properly colour $G^2[P]$.  To show that this is possible, we will prove that, with positive probability, after phase 1, the maximum out-degree of each vertex in $H[P]$ is at most $Cd\sqrt{\Delta}\log n$ for some constant $C$ that does not depend on $d$, $\Delta$, or $n$.  Therefore there is some assignment of colours in phase one in which the maximum out-degree in $H[P]$ is at most $Cd\sqrt{\Delta}\log n$.  Since $H$ is acyclic, this implies that $G^2[P]$ is $Cd\sqrt{\Delta}\log^{3/2} n$-degenerate, so the $Cd\sqrt{\Delta}\log^{3/2} n+1$ colours in $\Phi_{k+1}$ are sufficient to properly colour it.

  For each vertex $v$ of $H$, let $N_H(v):=\{v\}\cup \{y\in V(H): \overrightarrow{vy}\in E(H)\}$ be the set of out-neighbours of $v$ in $H$.  We now focus on a specific vertex $v$ in $G$ and show that, with probability $1-o(1/n)$, $|N_H(v)\cap P|\le Cd\sqrt{\Delta}\log^{3/2}  n$.  This shows that with probability $1-o(1)$, every vertex $v$ of $P$ has out-degree at most $Cd\sqrt{\Delta}\log^{3/2} n$ in $H[P]$.  In particular, it implies that there exists a colouring of $G$ with this property.

   Define $\outn(v):=N_H(v)\cap P$. For any vertex $w$ of $H$, let $\comp(v,w):=\{y\in N_H(v):\text{$w$ completes $y$}\}$.  Observe that
  \[
    \outn(v) = \bigcup_{w\in V(H)} \comp(v,w) = \bigcup_{w>v} \comp(v,w) \enspace .
  \]
  For any vertex $w>v$, let $C_{v,w}:=\{y\in N_H(v): yw\in E(G),\, y < w\}$, which is the set of vertices in $N_H(v)$ that $w$ could potentially complete.  For each $\alpha\in\Phi_w$, let $\comp'_\alpha(v,w)$ contain exactly those vertices $y\in C_{v,w}$ that have colour $\alpha$ or have a neighbour of colour $\alpha$ immediately before choosing the colour of $w$.  Observe that $\comp'_\alpha(v,w)$ contains every $y\in C_{v,w}$ that would be completed if we were to set the colour of $w$ to $\alpha$ (as well as some additional vertices that may have already been completed by vertices other than $w$).

  Recall that $N_\alpha(w)$ counts the number of neighbours $y$ of $w$ such that $y < w$ and $y$ has colour $\alpha$ or has a neighbour of colour $\alpha$ immediately before we choose the colour of $w$.  Therefore,  $|\comp'_\alpha(v,w)|\le N_\alpha(w)\le \sqrt{\Delta}$ for each $\alpha\in\Phi_w$.  Now let $\alpha\in\Phi_w$ be the colour that is actually chosen for $w$ and let $\comp'(v,w):=\comp'_\alpha(v,w)$.  We want to study the random variable $n'_{v,w}:=|\comp'(v,w)|\ge |\comp(v,w)|$.  Since $\alpha$ is chosen from $\Phi_w$, $n'_{v,w}\le\sqrt{\Delta}$ with probability $1$.

  Let $\alpha_1,\ldots,\alpha_p$ be the colours in $\Phi_w$ ordered so that,
  \[
    |\comp'_{\alpha_1}(v,w)|\ge|\comp'_{\alpha_2}(v,w)|\ge\cdots\ge |\comp'_{\alpha_p}(v,w)| \enspace .
  \]
  Immediately before colouring $w\in L_i$, each $y\in C_{v,w}$ is assigned a colour (possibly in $\Phi_i$) and has at most $4d-2$ neighbours that have already received a colour in $\Phi_i$.  Therefore, $y$ appears in
  $\comp'_{\alpha}(v,w)$ for at most $4d-1$ values of $\alpha$, so
  \[
    \sum_{\alpha\in\Phi_w} |\comp'_{\alpha}(v,w)| \le 4d| C_{v,w}| \enspace .
  \]
  Therefore $i|\comp'_{\alpha_i}(v,w)|\le\sum_{j=1}^i|\comp'_{\alpha_j}(v,w)|\le 4d|C_{v,w}|$, so
  \[
    |\comp'_{\alpha_i}(v,w)|\le 4d|C_{v,w}|/i \enspace .
  \]
  Therefore, regardless of any random choices made before choosing the colour of $w$ and any random choices made after choosing the colour of $w$, the random variable $n'_{v,w}$ is dominated by a random variable $N'_{v,w}:=\min\{|C_{v,w}|/j,\sqrt{\Delta}\}$ where $j$ is chosen uniformly in $\{1,\ldots,\Delta\}$.

  Therefore, $|\outn(v)|$ is dominated by a sum $X:=\sum_{w> v} N'_{v,w}$ of independent random variables.  In the appendix, we show how to use one of Bernstein's Inequalities to prove that $\Pr\left(X\ge C\sqrt{\Delta}\log^
  {3/2} n\right)\le n^{-\Omega(C^2/(d+C))} = n^{-\Omega(c')}$ for $C:=\sqrt{c'd}$.  Thus, the number of additional colours needed to recolour $P$ is $O(\sqrt{d\Delta}\log^{3/2} n)$.
\end{proof}

\section{Generalization to $\ell$-Vertex-Ranking}

A vertex colouring $\varphi$ of $G$ is ab \defin{$\ell$-vertex-ranking} of $G$ if, for each path $v_0,\ldots,v_r$ in $G$ with $1\le r\le\ell$ edges, $\varphi(v_0)\neq \varphi(v_r)$ or $\max\{\varphi(v_1),\ldots,\varphi(v_{r-1})\}>\varphi(v_0)$.  The \defin{$\ell$-vertex-ranking} number $\lrn(G)$ is the minimum integer $k$ such that $G$ has a vertex $\ell$-ranking $\varphi:V(G)\to\{1,\ldots,k\}$.  Note that $\varphi$ is a $2$-vertex ranking of $G$ if and only if it is a us-colouring of $G$, so $\trn(G)=\rn{2}(G)$ for every graph $G$.  The same proof gives the following generalization of \cref{d_degenerate_upper_bound}

\begin{thm}
  For any fixed integers $d,\ell\ge 1$, every $n$-vertex $d$-degenerate graph $G$ has $\lrn(G) \le O(n^{(\ell-1)/(\ell+1)}\log  n)$.
\end{thm}

\begin{proof}[Proof Sketch]
  Let $p:=n^{(\ell-1)/(\ell+1)}$ and let $\Delta=n^{2/(\ell+1)}$.  As in the proof of \cref{d_degenerate_upper_bound} eliminate all vertices of degree at last $\Delta$ using $n/\Delta = n^{(\ell-1)/(\ell+1)}=p$ additional high colours.  Next, proceed exactly as in the proof of \cref{degenerate_and_degree} except define $H$ as the directed version of $G^{\ell}$.  Observe that, for every directed edge $\overrightarrow{vy}$ of $H$, $G$ contains a path $v_0,\ldots,v_r$ with $v_0=v$, $v_r=y$, $r\le \ell$ such that $v_a\in L_i$ and $v_{a+1}\in L_j$ for some $i\le j$.  The number of such paths is $O(\Delta^{\ell-1})$. When colouring a vertex $w$ choose the colour of $w$ randomly from a $p$ element subset of a palette of $2p$ colours chosen.  Similar calculations show that, with high probability the number of problematic vertices in $N_H(v)$ is at most $O(|N_H(v)|\log n/p)=O(\Delta^{\ell-1}/p)=O(p\log n)$.  Thus, the subgraph of $G^\ell$ induced by the set of problematic vertices is $O(p\log n)$-degenerate and can be properly coloured with $O(p\log n)$ colours.
\end{proof}

\todo[inline]{There seems to be a link here with $r$-weak colouring numbers since we've fixed a total order and we're basically counting paths of length up to $\ell$ that start at $v$ and finish at some vertex that comes after $v$.  We should explore this.}

\todo[inline]{Is there an $\Omega(n^{(\ell-1)/(\ell+1)})$ lower bound on the $\ell$-vertex-ranking number of $d$-degenerate graphs for some small constant $d$?}

\bibliographystyle{plainnat}
\bibliography{us2}

\appendix

\section{Bounding the Tail of \texorpdfstring{\boldmath$|\outn(v)|$}{out(v)}}

We make use of the following inequality of Bernstein \cite[Corollary~2.11]{boucheron.lugosi.ea:concentration}:

\begin{thm}\label{bernstein_theorem}
  Let $k$ be a positive number, let $X_1,\ldots,X_n$ be independent random variables such that $\Pr(X_i\le k)=1$ for each $i\in\{1,\ldots,n\}$, and let $X:=\sum_{i=1}^n X_i$. Then
  \begin{equation}
    \Pr\left(X \ge \E(X)+ t\right)
      \le \exp\left(\frac{\tfrac{1}{2}t^2}{\sum_{i=1}^n \E((X_i-\E(X_i))^2)+\tfrac{1}{3}kt}\right) \enspace . \label{bernstein}
  \end{equation}
\end{thm}
We will apply \cref{bernstein_theorem} to random variables in which $X_i$ has the following distribution (for some $x_i\le k^2$):
\[
  X_i = \begin{cases}
          k & \text{with probability $x_i/k^2$} \\
          x_i/j & \text{with probability $1/k$ for $j\in\{x_i/k+1,\ldots,k\}$}
        \end{cases}
\]
This is the distribution we get when we choose a uniform $j\in\{1,\ldots,k\}$ and set $X_i:=\min\{k,x_i/j)$.
Then we have
\[
  \E(X_i)=\frac{x_i}{k^2}\cdot k + \frac{1}{k}\cdot\sum_{j=x_i/k+1}^k \frac{x_i}{j}
  = \frac{x_i}{k}(1+H_{k}-H_{x_i/k})
  \le \frac{x_i}{k}(1+H_{k})
  \le \frac{x_i(2+\ln k)}{k}
  = O\left(\frac{x_i\log k}{k}\right)
\]
so
\[
  (\E(X_i))^2 = O\left(\left(\frac{x_i\log k}{k}\right)^2\right) = O(x_i\log^2 k) \enspace ,
\]
since $x_i\le k^2$,
and
\begin{align*}
  \E((X_i-\E(X_i))^2)
  & = \frac{x_i}{k^2}\cdot(k-\E(X_i))^2
    + \frac{1}{k}\cdot\sum_{j=x_i/k+1}^k \left(\frac{x_i}{j}-\E(X_i)\right)^2 \\
  & = \frac{x_i}{k^2}\left(k^2-2k\E(X_i)+(\E(X_i))^2\right)
    + \frac{1}{k}\cdot \sum_{j=x_i/k+1}^k \left(\frac{x_i}{j}-\E(X_i)\right)^2 \\
  & \le \frac{x_i}{k^2}\left(k^2 + (\E(X_i))^2\right)
    + \frac{1}{k}\cdot \sum_{j=x_i/k+1}^k \left(\frac{x_i}{j}-\E(X_i)\right)^2 \\
  & = x_i + O\left(\frac{x_i^2\log^2 k}{k^2}\right)
      + \frac{1}{k}\cdot\sum_{j=x_i/k+1}^k \left(\frac{x_i}{j}-\E(X_i)\right)^2 \\
  & \le O(x_i\log^2 k)
      + \frac{1}{k}\cdot\sum_{j=x_i/k+1}^k \left(\frac{x_i}{j}-\E(X_i)\right)^2
      & \text{(since $x_i\le k^2$)} \\
 & = O(x_i\log^2 k) + \frac{1}{k}\cdot\sum_{j=x_i/k+1}^k \left(\frac{x_i^2}{j^2}-\frac{2x_i \E(X_i)}{j} + (\E(X_i))^2\right) \\
 & \le O(x_i\log^2 k) + \frac{1}{k}\cdot\sum_{j=x_i/k+1}^k \left(\frac{x_i^2}{j^2} + (\E(X_i))^2\right) \\
 & = O(x_i\log^2 k) +  \left(\frac{x_i^2}{k}\sum_{j=x_i/k+1}^k\frac{1}{j^2}\right)
   + \left(\sum_{j=x_i/k}^k O\left(\frac{x_i\log^2 k}{k}\right)\right) \\
 & \le O(x_i\log^2 k) + \left(\frac{x_i^2}{k}\sum_{j=x_i/k+1}^k\frac{1}{j^2}\right)
    + O(x_i\log^2 k)\\
  & \le O(x_i\log^2 k) + \frac{\pi^2 x_i}{6} + O(x_i\log^2 k) \\
  & = O(x_i\log^2 k)
\end{align*}
where the penultimate inequality follows from the inequality $\sum_{j=a}^{\infty} 1/j^2 \le \pi^2/(6a)$, valid for all integers $a\ge 1$.

To apply this in our setting, let $k:=\sqrt{\Delta}$, let $\{w_1,\ldots,w_n\}:=\{w\in V(H):w>v\}$, and let $x_i:=\min\{|C_{v,w_i}|,\Delta\}$.  We have the constraints $\sum_{i=1}^n x_i \le (4d-1)k^2 < 4d\Delta$.  Maximizing $\sum_{i=1}^n\E((X_i-\E(X_i))^2)=\sum_{i=1}^n O(x_i\log^2 x_i)$ subject to this constraint is easy and gives
\[
  \sum_{i=1}^n\E((X_i-\E(X_i))^2) = O(d\Delta\log^2 \Delta)
\]
Setting $t=C\sqrt{\Delta}\log^{3/2} n$ and putting this into \cref{bernstein} gives:
\begin{align*}
  \Pr(X\ge \E(X)+t)
  & = \Pr(X\ge \E(X)+C\sqrt{\Delta}\log^{3/2} n) \\
  & \le \exp \left(-\frac{\tfrac{1}{2}\left(\sqrt{\Delta}\log^{3/2} n\right)^2}{O(d\Delta\log^2\Delta) + \tfrac{1}{3}kt}\right) \\
  & \le \exp \left(-\frac{\tfrac{1}{2}C^2\Delta\log^3 n}{O(d\Delta\log^2\Delta) + \tfrac{1}{3}kt}\right) \\
  & \le \exp \left(-\frac{\tfrac{1}{2}C^2\Delta\log^2 n}{O(d\Delta\log^2\Delta) + \tfrac{1}{3}C\Delta\log^{3/2}\Delta}\right) & \text{(since $k=\sqrt{\Delta}$)} \\
  & \le \exp \left(-\frac{\tfrac{1}{2}C^2\Delta\log^2 n}{O(d\Delta\log^2n) + \tfrac{1}{3}C\Delta\log^{3/2}n}\right)
  & \text{(since $\Delta \le n$)} \\
  & \le \exp \left(-\Omega((C^2/(d+C))\log n)\right) \\
  & \le n^{-\Omega(C^2/(d+C))} \enspace .
\end{align*}

\end{document}

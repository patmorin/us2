\documentclass{patmorin}
% \usepackage[OT1]{fontenc}
% \usepackage[utf8]{inputenc}
\listfiles


\definecolor{brightmaroon}{rgb}{0.76, 0.13, 0.28}
\definecolor{linkblue}{rgb}{0, 0.337, 0.227}
\newcommand{\defin}[1]{\emph{\color{brightmaroon}#1}}
\setlength{\parskip}{1ex}

\DeclareMathOperator{\comp}{comp}

\title{\MakeUppercase{Vertex Ranking of Degenerate Graphs}}
 % \newline or \MakeUppercase{Vertex Ranking for Degenerates}}
\author{%
  John Iacono%
    \thanks{Université libre de Bruxelles,
      Department of Computer Science, Algorithms Research Group} \and
  Piotr Micek%
    \thanks{Jagiellonian University,
      Faculty of Mathematics and Computer Science,
      Theoretical Computer Science Department} \and
  Pat Morin%
    \thanks{Carleton University, School of Computer Science}\and
  Bruce Reed%
    \thanks{Academia Sinica, Institute of Mathematics}}


\newcommand{\rn}[1]{\chi_{\operatorname{#1-vr}}}
\newcommand{\irn}{\rn{\infty}}
% \newcommand{\trn}{\rn{2}}
\newcommand{\trn}{\chi_{\mathrm{us}}}
\newcommand{\lrn}{\rn{\ell}}
\newcommand{\dtcn}{\bar{\chi}_2}
\newcommand{\dlcn}{\bar{\chi}_\ell}
\newcommand{\scn}{\chi_{\star}}

\newcommand{\pat}[1]{\textcolor{red}{PM:#1}}

\newcommand{\texp}{1-1/(\lfloor\ell/2\rfloor+1/2)}
\newcommand{\dexp}{1-\frac{1}{\lfloor\ell/2\rfloor+1/2}}

% \pagenumbering{roman}
\begin{document}

\maketitle

\begin{abstract}
  An $\ell$-vertex-ranking of a graph $G$ is a colouring of the vertices of $G$ with integer colours so that in any connected subgraph $H$ of $G$ having diameter at most $\ell$ there is a vertex in $H$ whose colour is larger than that of every other vertex in $H$.  The $\ell$-vertex-ranking number, $\lrn(G)$, of $G$ is the minimum integer $k$ such that $G$ has an $\ell$-vertex-ranking using $k$ colours.  We prove that, for any fixed $d$ and $\ell$, every $d$-degenerate $n$-vertex graph $G$ satisfies $\lrn(G)= O(n^{1-2/(\ell+1)}\log n)$ if $\ell$ is even and $\lrn(G)= O(n^{1-2/\ell}\log n)$ if $\ell$ is odd. The case $\ell=2$ resolves (up to the $\log n$ factor) an open problem posed by \citet{karpas.neiman.ea:on} and the cases $\ell\in\{2,3\}$ are asymptotically optimal (up to the $\log n$ factor).
\end{abstract}


\section{Introduction}

An $\ell$-vertex-ranking of a graph $G$ is a colouring of the vertices of $G$ with integer colours so that in any connected subgraph $H$ of $G$ having diameter at most $\ell$ there is a vertex in $H$ whose colour is larger than that of every other vertex in $H$.  The \defin{$\ell$-vertex-ranking} number $\lrn(G)$ is the minimum integer $k$ such that $G$ has a $\ell$-vertex-ranking $\varphi:V(G)\to\{1,\ldots,k\}$. The \defin{$\ell$-vertex-ranking} number of a graph class $\mathcal{G}$ is $\lrn(\mathcal{G}):=\sup\{\lrn(G):G\in\mathcal{G}\}$.



When $\ell=1$, a colouring is a $1$-vertex-ranking if and only if it is a proper colouring of $G$, so $\chi(G)=\rn{1}(G)$ and, for any $\ell\ge 1$, $\chi(G)\le \lrn(G)$.  Besides the case $\ell=1$, two special cases have received extra attention.  When $\ell=\infty$, $\rn{\infty}(G)$ is equal to the \defin{vertex ranking number}, the \defin{centered chromatic number}, and the \defin{treedepth} of $G$
\cite{nesetril.ossona:tree-depth},
which plays a central role in the theory of sparsity  \cite{nesetril.ossona:sparsity}.
At the other extreme, when $\ell=2$, a $2$-vertex-ranking of $G$ is also known as a \defin{restricted star colouring} \cite{shalu.antony:complexity} or a \defin{unique superior colouring} of $G$ \cite{karpas.neiman.ea:on}.  Previous results for $\ell=2$ and fixed $\ell\ge 2$ are summarized in \cref{previous_works}.

\begin{table}
    \centering{
        \begin{tabular}{|l|c|c|l|} \hline
          \multicolumn{4}{|c|}{$\rn{2}$} \\\hline
            Graph class & Upper Bound & Lower Bound & Ref. \\ \hline
            Trees & $O(\log n/\log\log n)$ & $\Omega(\log n/\log\log n)$ & \cite{karpas.neiman.ea:on} \\
            Planar graphs & $O(\log n/\log^{(3)} n)$ & $O(\log n/\log^{(3)} n)$ & \cite{bose.dujmovic.ea:asymptotically} \\
            Proper minor closed & $O(\log n)$ & $\Omega(\log n/\log\log n)$ & \cite{karpas.neiman.ea:on} \\
            $d$-cubes & $d+1$ & $d+1$ & \cite{almeter.demircan.ea:graph} \\
            Max-degree 3 & $7$ & & \cite{almeter.demircan.ea:graph} \\
            Max-degree $\Delta$ & $O(\min\{\Delta^2,\Delta\sqrt{n}\})$ & $\Omega(\Delta^2/\log \Delta)$ & \cite{karpas.neiman.ea:on,almeter.demircan.ea:graph} \\
            $d$-degenerate & $O(d\sqrt{n})$ & $\Omega(n^{1/3} + d^2/\log d)$ & \cite{karpas.neiman.ea:on,almeter.demircan.ea:graph} \\
            \hline \multicolumn{4}{c}{} \\
            \hline
            \multicolumn{4}{|c|}{$\rn{\ell}$, for fixed $\ell\ge 2$} \\\hline
            Simple treewidth $\le t$ & $O(\log n/\log^{(t)} n)$ & $\Omega(\log n/\log^{(t)} n)$ & \cite{bose.dujmovic.ea:asymptotically} \\
            Treewidth $\le t$ & $O(\log n/\log^{(t+1)} n)$ & $\Omega(\log n/\log^{(t+1)} n)$ & \cite{bose.dujmovic.ea:asymptotically} \\
            Planar graphs & $O(\log n/\log^{(3)} n)$ & $\Omega(\log n/\log^{(3)} n)$ & \cite{bose.dujmovic.ea:asymptotically} \\
            Outerplanar graphs & $O(\log n/\log^{(2)} n)$ & $\Omega(\log n/\log^{(2)} n)$ &  \cite{bose.dujmovic.ea:asymptotically,karpas.neiman.ea:on} \\
            Genus-$g$ graphs & $O(g\log n/\log^{(3)} n)$ & $\Omega(\log n/\log^{(3)} n)$ & \cite{bose.dujmovic.ea:asymptotically} \\
            % $A$-minor-free (apex $A$) & $O(\log n/\log^{(c(A))} n)$ & $\Uparrow$ & \cref{meta} \\
            % $(g,k)$-planar & $O(\log n/\log^{(c(g,k))} n)$ & $\Uparrow$  & \cref{meta} \\
            \hline
        \end{tabular}
    } % centering
    \caption{Summary of previous results on $\rn{2}$ and $\lrn$ for fixed $\ell\ge 2$. Here, $\log^{(c)} n:=\underbrace{\log\log\cdots\log}_c n$.
    }
\label{previous_works}
\end{table}

The current work is motivated by the gap between the upper and lower bounds for $\rn{2}(G)$ for $d$-degenerate graphs $G$. For fixed $d\ge 2$, the upper bound is $O(\sqrt{n})$, while the lower bound is $\Omega(n^{1/3})$. Closing this gap is stated explicitly as an open problem by \citet{karpas.neiman.ea:on} and \citet{bose.dujmovic.ea:asymptotically}. Our first contribution solves
this problem, up to a logarithmic factor.

\begin{thm}\label{unique_superior}
  For any positive integer $d$ there exists a constant $c:=c(d)$ such that, for every integer $n\ge d$, every $n$-vertex $d$-degenerate graph $G$ satisfies $\rn{2}(G)\le cn^{1/3}\log n$.
\end{thm}

% Note that $\varphi$ is a $2$-vertex ranking of $G$ if and only if it is a unique-superior colouring of $G$, so $\trn(G)=\rn{2}(G)$ for every graph $G$.
\Cref{unique_superior} is an immediate consequence of the following more general upper bound for the $\ell$-vertex-ranking number of $d$-degenerate graphs.

\begin{thm}\label{l_d_degenerate_upper_bound}
  For any positive integers $d$ and $\ell$ there exists a constant $c:=c(d,\ell)$ such that, for every integer $n\ge d$, every $n$-vertex $d$-degenerate graph $G$ satisfies
  \[
    \lrn(G)\le c n^{\dexp}\log n
    = \begin{cases}
      cn^{1-\frac{2}{\ell}}\log n & \text{if $\ell$ is odd} \\
      cn^{1-\frac{2}{\ell+1}}\log n & \text{if $\ell$ is even.}
      \end{cases}
  \]
\end{thm}
\Cref{l_d_degenerate_upper_bound} exhibits a parity phenomenon one encounters when counting the number of paths of length $\ell$ in a $d$-degenerate graph of maximum-degree $\Delta$. Because of this, the bound in \cref{l_d_degenerate_upper_bound} is the same for $\ell=2k$ and $\ell=2k+1$, for any positive integer $k$.  One might think that this is just an artifact of the proof technique and that the bound for even values of $\ell$ is not tight.  However, \citet{karpas.neiman.ea:on} proved the existence of $2$-degenerate $n$-vertex graphs with $\rn{2}(G)\in\Omega(n^{1/3})$ and \cref{l_d_degenerate_upper_bound} (with $\ell=2$) matches this lower bound to within a logarithmic factor.  Since $\rn{3}(G)\ge\rn{2}(G)$ for any graph $G$, it also matches this bound when $\ell=3$.  Thus, \cref{l_d_degenerate_upper_bound} is tight (up to a $\log n$ factor) for $\ell=2$ and $\ell=3$, leading us to suspect that it is tight for any fixed $\ell$.


Like the upper bound on $\rn{2}(G)$ in \cite{karpas.neiman.ea:on}, \cref{l_d_degenerate_upper_bound} follows quickly from a theorem about graphs that are both $d$-degenerate and have maximum-degree $\Delta$. For such graphs, we prove:

\begin{thm}\label{l_degenerate_and_degree}
  For all positive integers $d$ and $\ell\ge 2$ there exists a constant $c:=c(d,\ell)$ such that, for every integer $\Delta\ge d$ and every integer $n\ge \Delta$, every $n$-vertex $d$-degenerate graph $G$ of maximum-degree at most $\Delta$ satisfies
  \[
    \lrn(G)\le c\Delta^{\lfloor\ell/2\rfloor-1/2}\log^{5/4} n
    =\begin{cases}
    c\Delta^{\ell/2-1}\log^{5/4} n & \text{if $\ell$ is odd.} \\
    c\Delta^{\ell/2-1/2}\log^{5/4} n & \text{if $\ell$ is even.}
    \end{cases}
  \]
  Furthermore, if $\Delta^{\lfloor\ell/2\rfloor-1}\ge\log n$ then $\lrn(G)\le c\Delta^{\lfloor\ell/2\rfloor-1/2}\log n$.
  % \[
  %    \lrn(G) \le \begin{cases}
  %                   c\Delta^{\lfloor\ell/2\rfloor-1/2}\log^{5/4} n
  %                     & \text{for $\ell\in\{2,3\}$} \\
  %                   c\Delta^{\lfloor\ell/2\rfloor-1/2}\log n
  %                     & \text{for $\ell\ge 4}
  %                 \end{cases}
  % \]
\end{thm}

\Cref{l_d_degenerate_upper_bound} follows from \cref{l_degenerate_and_degree}, by the following easy argument:  Let $S$ be the set of vertices in $G$ having degree at least $\Delta$, for some carefully chosen value of $\Delta$.  Since $G$ is $d$-degenerate, it has at most $dn$ edges, so the total degree of all vertices in $G$ is at most $2dn$, so $|S|\le 2dn/\Delta$.  Now apply \cref{l_degenerate_and_degree} to the graph $G-S$ which is $d$-degenerate and has maximum-degree $\Delta$ to obtain a colouring $\varphi:V(G-S)\to\{1,\ldots,k\}$.  Finally, colour every vertex in $S$ with a distinct colour larger than $k$.  In this way, the total number of colours used is $k + |S|\le c\Delta^{\lfloor\ell/2\rfloor-1/2}\log^{5/4} n + 2dn/\Delta$. Choosing $\Delta$ to balance these two quantities yields \cref{l_d_degenerate_upper_bound}.  This argument is presented in a little more detail at the end of \cref{the_proof}.






% The case $\ell=2$ resolves an open problem of \citet{karpas.neiman.ea:on}.

% \todo[inline]{Does this look better?
% \[ c n^{(\lceil\ell/2\rceil-1/2)/(\lceil\ell/2\rceil+1/2)}\log n \]
% % \[ c \exp\left((\lceil\ell/2\rceil-1/2)/(\lceil\ell/2\rceil+1/2)\right)\log n \]
% \[ cn^{1-\frac{1}{\lceil\ell/2\rceil+1/2}}\log n \]
% \[ cn^{1-\frac{1}{\lceil\ell/2\rceil+1/2}}\log n \]
% In text: $cn^{1-1/(\ell-1/2)}\log n$ versus $cn^{(\ell-3/2)/(\ell-1/2)}\log n$
% }

% In the next section, we prove \cref{degenerate_and_degree}.  In the subsequent section we sketch the modifications needed to prove \cref{l_degenerate_and_degree}, the $\ell$-vertex-ranking analogue of  \cref{degenerate_and_degree} and use this to establish \cref{l_d_degenerate_upper_bound}.



\section{Preliminaries}

For any standard graph-theoretic terminology and notation not defined here, we use the same conventions used in the textbook by \citet{diestel:graph}.  A graph $G$ has vertex set $V(G)$ and edge set $E(G)$.  For any $S\subseteq V(G)$, $G[S]$ denote the subgraph of $G$ induced by the vertices in $S$.  For any vertex $v$ of $G$, $N_G(v):=\{w:vw\in E(G)\}$ and $\deg_G(v):=|N_G(v)|$.  For an integer $\ell$, $G^\ell$ denotes the graph with vertex set $V(G)$ that contains an edge $vw$ if and only if some path in $G$ with at most $\ell$ edges contains both $v$ and $w$.

The following alternative definition of $\ell$-vertex-ranking turns out to be more convenient for proofs and is what we will use from this point on.
\begin{obs}\label{alternate_def}
  A vertex colouring $\varphi:V(G)\to\{1,\ldots,k\}$ of a graph $G$ is an $\ell$-vertex ranking of $G$ if and only if, for each path $v_0,\ldots,v_r$ in $G$ with at most $\ell$ edges, $\varphi(v_0)\neq \varphi(v_r)$ or $\max\{\varphi(v_1),\ldots,\varphi(v_{r-1})\}>\varphi(v_0)$.
\end{obs}

For a directed graph $G$, we write $\overrightarrow{vw}$ to denote the directed edge with source $v$ and target $w$.  For a vertex $v$ in a directed graph $G$, $N^+_{G}(v):=\{w\in V(G):\overrightarrow{vw}\in E(G)\}$ denotes the set of out-neighbours of $v$ and $N^-_G(v):=\{u\in V(G):\overrightarrow{uv}\in E(G)\}$ denotes the set of in-neighbours of $v$, $\deg^+_{G}(v):=|N^+_G(v)|$ is the out-degree of $v$, and $\deg^-_{G}(v):=|N^-_G(v)|$ is the in-degree of $v$. We also define $N^+_{G}[v]:=\{v\}\cup N^+_{G}(v)$ and $N^-_{G}[v]:=\{v\}\cup N^-_{G}(v)$ to be the closed out- and in-neighbourhoods of $v$, respectively.

We repeatedly make use of the following foklore result:

\begin{obs}\label{orientation_to_degeneracy}
  If an undirected graph $G$ has an orientation in which each vertex has out-degree at most $d$, then $G$ is $2d$-degenerate.
\end{obs}

\begin{proof}
  For any $S\subseteq V(G)$, the orientation shows that the induced subgraph $G[S]$ contains at most $d|S|$ edges and therefore the total degree of all vertices in $G[S]$ is at most $2d|S|$.  Therefore, for any $S\subseteq V(G)$, the induced graph $G[S]$ has a vertex of degree at most $2d$.
\end{proof}

An \defin{undirected path} $\Pi$ in a graph $G$ is a tree with exactly two leaves whose edges are all edges of $G$. The leaves of $\Pi$ are called the \defin{endpoints} of $\Pi$. The \defin{length} of $\Pi$ is the number of edges in $\Pi$, which is exactly one less than the number of vertices.  With a slight abuse of notation, we write $x_0,\ldots,x_r$ to denote a length-$r$ undirected path $\Pi$ where $x_0$ and $x_r$ are the endpoints of $\Pi$ and $\Pi$ contains the edge $x_{i-1}x_i$ for each $i\in\{1,\ldots,r\}$.  Note that each undirected path $\Pi:=x_0,\ldots,x_r$ in $G$ corresponds to exactly two paths $x_0,x_1,\ldots,x_r$ and $x_r,x_{r-1},\ldots,x_0$ in $G$.

For a graph $G$, let $\mathcal{P}_\ell(G)$ denote the set of all undirected paths of length at most $\ell$ in $G$.  The set $\mathcal{P}_\ell(G)$ is critical for us since, by \cref{alternate_def}, this is precisely the set of paths that need to be considered to determine if a vertex-colouring $\varphi$ of $G$ is an $\ell$-vertex-ranking.  The following lemma shows that the paths in $\mathcal{P}_\ell(G)$ can be mapped onto their endpoints in such a way that no endpoint receives too many paths.  Its proof uses a technique introduced by Cairns to upper bound the number of length-$\ell$ paths in planar graphs (see also \cite[Lemma~5]{devroye.dujmovic.ea:notes}).

\begin{lem}\label{advanced_cairns}
  For any integers $d\ge 2$, $\ell\ge 2$, $\Delta\ge d$ and any
  graph $G$ of maximum-degree $\Delta$ that has an orientation of maximum out-degree $d$ there exists a mapping $\rho:\mathcal{P}_\ell(G)\to V(G)$ such that
  \begin{compactenum}[(i)]
    \item $\rho(\Pi)$ is an endpoint of $\Pi$ for each $\Pi\in\mathcal{P}_\ell(G)$; and
    \item $|\rho^{-1}(v)| \le 2^{\ell+1}d^{\lceil \ell/2\rceil}\Delta^{\lfloor\ell/2\rfloor}$ for each $v\in V(G)$.
  \end{compactenum}
\end{lem}

\begin{proof}
  Fix some orientation of $G$ of maximum out-degree $d$. For each $\Pi\in\mathcal{P}_\ell(G)$ let $x_0,\ldots,x_r$ be one of the two paths in $G$ that corresponds to $\Pi$, chosen so that the edge $x_{i-1}x_{i}$ is oriented from away from $x_{i-1}$ and towards $x_{i}$ for at least half the indices $i\in\{1,\ldots,r\}$.  When $x_{i-1}x_{i}$ is oriented away from $x_{i-1}$, we call it a \defin{downstream edge} of $\Pi$. Otherwise we call $x_{i-1}x_{i}$ an \defin{upstream edge} of $\Pi$. In other words, we choose the endpoint $x_0$ so that at least half the edges of $\Pi$ are downstream edges, and we set $\rho(\Pi):=x_0$.  Observe that the path $x_0,\ldots,x_r$ can be uniquely reconstructed from the following information:
  \begin{compactenum}[(a)]
    \item A sequence $b_1,\ldots,b_r$ of $r$ bits, where $b_i=1$ if $x_{i-1}x_i$ is a downstream edge of $\Pi$ and $b_i=0$ if $x_{i-1}x_i$ is an upstream edge of $\Pi$.\label{bitstring}
    \item A sequence $\delta_1,\ldots,\delta_r$ of integers, where $\delta_i\in\{1,\ldots,d\}$ if $b_i=1$ and $\delta_i\in\{1,\ldots,\Delta\}$ if $b_i=0$.  The integer $\delta_i$ uniquely identifies the neighbour $x_i$ of $x_{i-1}$ so that, starting at $x_0$ we can uniquely reconstruct the path $x_0,\ldots,x_r$ which uniquely identifies the undirected path $\Pi$.\label{directions}
  \end{compactenum}
  The number of choices for (\ref{bitstring}) is $2^r$.  Since (\ref{bitstring}) has at least as many $1$-bits as $0$-bits, the number of choices for (\ref{directions}) is at most $d^{\lceil r/2\rceil}\Delta^{\lfloor r/2\rfloor}$.  So the number of paths of length $r$ in $\rho^{-1}(x_0)$ is at most $2^rd^{\lceil r/2\rceil}\Delta^{\lfloor r/2\rfloor}$.  Summing $r$ over $1$ to $\ell$ completes the proof.
\end{proof}

Observe that, for each edge $vw$ in $G^{\ell}$ there is at least one path in $\mathcal{P}_\ell(G)$ with endpoints $v$ and $w$.  For each edge $vw$ of $G^{\ell}$ we can select one such representative path $\Pi_{vw}\in\mathcal{P}_\ell(G)$.  If we then orient each edge $vw$ of $G^{\ell}$ away from $\rho(\Pi_{vw})$ then we get an orientation of $G^{\ell}$ in which each vertex has out-degree at most $2^{\ell+1}d^{\lceil \ell/2\rceil}\Delta^{\lfloor\ell/2\rfloor}$.  Combined with \cref{orientation_to_degeneracy} this gives the following corollary:

\begin{cor}\label{degeneracy_of_g_l}
  For any integers $d\ge 1$, $\Delta\ge d$ and any $d$-degenerate graph $G$ of maximum degree $\Delta$, $G^{\ell}$ is $2^{\ell+2}d^{\lceil \ell/2\rceil}\Delta^{\lfloor\ell/2\rfloor}$-degenerate.
\end{cor}

In order to obtain the bound in \cref{l_degenerate_and_degree} we need a special version of \cref{advanced_cairns} that only considers undirected paths $v_0,\ldots,v_r\in \mathcal{P}_\ell(G)$ such that $v_0v_1$ is directed toward $v_0$ and $v_{r-1}v_r$ is directed toward $v_r$.  For a directed graph $G'$ whose underlying undirected graph is $G$, let $\widehat{\mathcal{P}}_\ell(G')$ denote the set of undirected paths $x_0,\ldots,x_r$ in $G$ of length $r\le\ell$ and such that $\overleftarrow{x_0x_1}$ is an edge of $G'$ and $\overrightarrow{x_{r-1}x_{r}}$ is also an edge of $G'$.

\begin{lem}\label{advanced_cairns2}
  For any integers $d\ge 2$, $\ell\ge 3$, $\Delta\ge d$ and any
  directed graph $G'$ of maximum degree $\Delta$ and maximum out-degree $d$, there exists a mapping $\gamma:\widehat{\mathcal{P}}_\ell(G')\to V(G')$ such that
  \begin{compactenum}[(i)]
    \item $\gamma(\Pi)\in V(\Pi)$ for each $\Pi\in\mathcal{P}$; and
    \item $|\gamma^{-1}(v)| \le 2^{\ell}d^{\lceil \ell/2\rceil+1}\Delta^{\lfloor\ell/2\rfloor-1}$ for each $v\in V(G')$.
  \end{compactenum}
\end{lem}

\begin{proof}
  The proof is almost identical to \cref{advanced_cairns2} with two modifications.  If $G'$ contains both edges $\overrightarrow{vw}$ and $\overrightarrow{wv}$ then this edge is considered as a downstream edge no matter which direction it is traversed, since there are at most $d$ options for edges leaving $v$ (one of which is $w$) and at most $d$ options for edges leaving $w$ (one of which is $v$).

  For an undirected path $\Pi:=v_0,\ldots,v_r\in\widehat{\mathcal{P}}_\ell(G')$, the total number of upstream edges in the path $v_1,\ldots,v_r$ and in the path $v_{r-1},\ldots,v_0$ is at most $r-2$. (The edge $v_{r-1}v_r$ is a downstream edge in the first path, $v_1v_0$ is a downstream edge in the second path, and each of the edges in $v_1,\ldots,v_r$ is upstream in at most one of the two paths.)  Therefore we can choose the endpoint $v_0$ so that the number of upstream edges in $v_1,\ldots,v_r$ is at most $\lfloor (r-2)/2\rfloor=\lfloor r/2\rfloor-1$ and set $\gamma(\Pi):=v_1$. For $r\ge 3$, $\Pi$ can then be obtained by taking the union of the paths $x_1,x_0$ and $x_1,x_2,\ldots,x_r$.  The first path consists of one downstream edge, so there are at most $d$ options for the first path.  The second path has $r-1$ edges and at most $\lfloor r/2\rfloor-1$ of these are upstream edges, so there are at most $2^{r-1}d^{\lceil r/2\rceil}\Delta^{\lfloor r/2\rfloor-1}$ options for the second path.
   % can be described by a path of length $1$ with one downstream edge (for which there are $d$ options) and a path of length $r-1$ with at most $\lfloor (r-2)/2\rfloor=\lfloor r/2\rfloor-1$ upstream edges (for which there are $2^{r-1}d^{\lceil r/2\rceil+1}\Delta^{\lfloor r/2\rfloor-1}$ options).
   Thus, the number of paths of length $r$ assigned to any vertex is at most $2^{r-1}d^{\lceil r/2\rceil+1}\Delta^{\lfloor r/2\rfloor-1}$ for $r\ge 3$ (and at most $d^r$ for $r\in\{1,2\}$). Again, the proof finish by summing over $r$ in $1,\ldots.\ell$.
\end{proof}


\section{The Proof}
\label{the_proof}

For a vertex colouring $\varphi:V(G)\to\N$ of a graph $G$, we say that an undirected path $\Pi:=x_0,\ldots,x_r$ in $G$ is an \defin{$\ell$-violation} of $\varphi$ if $\Pi$ has length $r\le\ell$ and $\varphi(x_0)=\varphi(x_r)=\max\{\varphi(x_0),\ldots,\varphi(x_r)\}$.  Observe that $\varphi$ is an $\ell$-vertex-ranking if and only if $G$ contains no $\ell$-violations of $\varphi$.

% Before proving \cref{l_degenerate_and_degree}, it is worth noting that \cref{degeneracy_of_g_l} already implies that for a graph $G$ on which \cref{l_degenerate_and_degree} applies, $G^\ell$ is $O(\Delta^{\lfloor\ell/2\rfloor})$-degenerate, so $\lrn(G)\le \chi(G^{\ell})\in O(\Delta^{\lfloor \ell/2\rfloor})$.  For $\Delta\ge \log^{5/2} n$,  \cref{l_degenerate_and_degree} reduces this bound by a factor of $\Delta^{1/2}\log^{-5/4} n$.

% \begin{thm}\label{l_degenerate_and_degree}
%   For any integers $d\ge 2$ and $\ell\ge 2$ there exists a constant $c:=c(d,\ell)$ such that, for all integers $\Delta\ge d$ and $n\ge \Delta$, every $n$-vertex $d$-degenerate graph $G$ of maximum degree $\Delta$ has $\lrn(G)\le c\Delta^{\lfloor\ell/2\rfloor-1/2}\log^{1+b} n$.
% \end{thm}

\begin{proof}[Proof of \cref{l_degenerate_and_degree}]
  Let $G$ be an $n$-vertex $d$-degenerate graph of maximum-degree $\Delta$.  Let $S_0:=V(G)$ and, for each integer $i\ge 1$, let $S_i:=\{v\in S_{i-1}:\deg_{G[S_{i-1}]}(v)\ge 4d\}$.  Since $G$ is $d$-degenerate $G[S_{i-1}]$ has at most $d|S_{i-1}|$ edges.  Therefore
  \[
    2d|S_{i-1}|\ge \sum_{v\in S_{i-1}} \deg_{G[S_{i-1}]}(v)\ge 4d|S_i| \enspace ,
  \]
  so $|S_i|\le |S_{i-1}|/2\le n/2^i$ for each $i\ge 1$.  Let $q$ be the maximum integer such that $S_q$ is non-empty.  Since $1\le |S_q|\le n/2^q$, $q\le \log_2 n$.  For each $i\in\{0,\ldots,q\}$, let $L_i:=S_i\setminus S_{i+1}$.  (These notations are mnemonics: $S_i$ contains the \defin{survivors} of round $i-1$ and $L_i$ is \defin{layer} $i$.)

  Let $b=1/4$ and let $k:=\Delta^{\lfloor\ell/2\rfloor-1/2}\log^b n$, so that our goal is to find an $\ell$-vertex-ranking of $G$ using $O(k\log n)$ colours. We compute our colouring using a two phase algorithm. In the first phase we use a sequence of pairwise-disjoint colour palettes $\Phi_0,\ldots,\Phi_{q}$, each of size $2k$, such that for each $1\le i < j\le q$, every colour in $\Phi_i$ is less than every colour in $\Phi_j$.  We will use the colours in $\Phi_i$ to colour the vertices in $L_i$, for each $i\in\{0,\ldots,q\}$.  The total number of colours used in this phase is $2k(q+1)\le 2k(1+\log n)= O(k\log n)$.  The first phase colouring $\varphi:V(G)\to\Phi_0\cup\cdots\cup\Phi_q$ may have some $\ell$-violations that will be eliminated by re-colouring some vertices in the second phase using an additional palette $\Phi_{q+1}$ of size $O(k\log n)$.

  Let $\mathcal{P}$ contain all the undirected paths $x_0,\ldots,x_r$ in $\mathcal{P}_{\ell}(G)$ such that $x_0$ and $x_r$ are in the same layer $L_j$ and $\{x_1,\ldots,x_{r-1}\}\subseteq \bigcup_{i=0}^j L_i$.  Since $G$ is $d$-degenerate, it has an orientation of maximum out-degree $d$.  Let $\rho:\mathcal{P}\to V(G)$ be the mapping given by \cref{advanced_cairns}, applied to $G$, restricted to the paths in $\mathcal{P}$. (The purpose of the restriction is so that $\rho^{-1}(v)$ denotes a subset of $\mathcal{P}$.)  We will say that an undirected path $\Pi:=x_0,\ldots,x_r$ in $\mathcal{P}$ is \defin{problematic} if $x_0$ and $x_r$ are assigned the same colour in the first phase of the algorithm.

  Although we have not yet completed the description of the first phase colouring procedure, we already know enough to establish the following claim:  Any $\ell$-violation $\Pi$ of the first phase colouring $\varphi$ is a problematic path in $\mathcal{P}$.  Indeed, if $\Pi=x_0,\ldots,x_r$ is an $\ell$-violation, then $\varphi(x_0)=\varphi(x_r)\in \Phi_j$ for some $j\in\{0,\ldots,q\}$.  Thus $x_0,x_r\in L_j$ for some $j\in\{0,\ldots,q\}$.  Furthermore, since $\Pi$ is an $\ell$-violation $\varphi(x_0)=\max\{\varphi(x_0),\ldots,\varphi(x_r)\}$, so no colour in $\Phi_{j+1},\ldots,\Phi_q$ appears at any vertex in $x_1,\ldots,x_{r-1}$.  Therefore, $\{x_1,\ldots,x_{r-1}\}\subseteq \bigcup_{i=0}^j L_i$.  Therefore $\Pi\in\mathcal{P}$.  Since $\varphi(x_0)=\varphi(x_r)$, $\Pi$ is problematic.

  We now describe the first phase colouring algorithm.  Consider the multigraph $G^*$ that, for each $vw\in V(G)$ contains as many copies of the edge $vw$ as there are undirected paths in $\mathcal{P}$ with endpoints $v$ and $w$.  The existence of $\rho$ implies that $G^*$ has an orientation in which each vertex has out-degree $O(\Delta^{\lfloor\ell /2\rfloor})$.  By \cref{orientation_to_degeneracy}, $G^*$ is $O(\Delta^{\lfloor\ell /2\rfloor})$-degenerate.  Label the vertices of $G$ as $v_1,\ldots,v_n$ so that $v_i$ has degree $O(\Delta^{\lfloor\ell /2\rfloor})$ in $G^*[\{v_1,\ldots,v_{i}\}]$. In other words, for each $a\in\{1,\ldots,n\}$ the number of paths in $\mathcal{P}$ with one endpoint at $v_a$ and the other endpoint in $\{v_1,\ldots,v_{a-1}\}$ is $O(\Delta^{\lfloor\ell /2\rfloor})$.  In the first phase, we will colour the vertices one by one in the order $v_1,\ldots,v_n$.

  Let $\tau:\mathcal{P}\to V(G)$ be the mapping that maps $\Pi\in\mathcal{P}$ to $v_b$ if and only if the endpoints of $\Pi$ are $v_a$ and $v_b$ and $a < b$.\footnote{The two mappings $\tau$ and $\rho$ are similar. The difference is that $\rho$ corresponds to some orientation of $G^*$ and $\tau$ corresponds to an acyclic orientation of $G^*$.}  Consider some vertex $w\in L_j$.  For each $\alpha \in \Phi_j$, let $N_{\alpha}(w)$ be the number of paths in $\tau^{-1}(w)$ whose other endpoint (not $w$) is assigned the colour $\alpha$. If $w=v_b$ then $N_{\alpha}(w)$ is completely determined by the colours of $v_1,\ldots,v_{b-1}$, so the value of $N_\alpha(v_b)$ is determined after $v_{b-1}$ is coloured but before $v_b$ is coloured.  Observe that assigning the colour $\alpha$ to $w$ creates exactly $N_{\alpha}(w)$ problematic paths, and these are all in $\tau^{-1}(w)$.  Therefore, $\sum_{\alpha\in\Phi_j} N_\alpha(w)=|\tau^{-1}(w)|\in O(\Delta^{\lfloor\ell /2\rfloor})$.

  For each vertex $w$, we choose the colour of $w$ uniformly at random from a subpalette of $\Phi_i$ that contains exactly half of the $2k$ colours in $\Phi_i$.  Specifically, we choose the colour of $w$ from a palette $\Phi(w)\subset \Phi_i$ that contains $k$ colours $\alpha$ in $\Phi_i$ with the smallest $N_\alpha(w)$ values, so that  $\max\{N_\alpha(w):\alpha\in \Phi(w)\}\le\min\{N_\alpha(w):\alpha\in\Phi_i\setminus\Phi(w)\}$.  Let $M:=c\Delta^{\lfloor\ell/2\rfloor}/k$ with $c$ sufficiently large so that $Mk\ge \tau^{-1}(w)$.
  Then
  \[
    Mk \ge |\tau^{-1}(w)|= \sum_{\alpha\in \Phi_i} N_\alpha(w) \ge \sum_{\alpha\in\Phi_i\setminus\Phi(w)}N_\alpha(w) \ge k\max\{N_\alpha(w):\alpha\in \Phi(w)\} \enspace .
  \]
  Therefore, $\max\{N_\alpha(w):\alpha\in \Phi(w)\}\le M$ is the maximum number of problematic paths in $\mathcal{P}$ that can be created by colouring $w$, and $M$ is the maximum number of problematic paths that can be created by colouring any single vertex.  This completes the description of the first-phase colouring $\varphi$ of $G$.


  The first phase colouring $\varphi$ is not yet an $\ell$-vertex-ranking.  Our goal now is to study the problematic paths in $\mathcal{P}$, each of which is a potential $\ell$-violation of $\varphi$.  For each problematic $\Pi\in\mathcal{P}$, we will choose a vertex $y$ of $\Pi$ to recolour in the second round to eliminate the possibility that $\Pi$ is an $\ell$-violation. of $\varphi$.  Consider the directed graph $G'$ obtained from $G$ by adding each edge $\overrightarrow{vw}$ if $vw\in E(G)$, $v\in L_i$, $w\in L_j$ and $i\le j$.  (Note that if $v$ and $w$ are in the same layer $L_j$ then both edges $\overrightarrow{vw}$ and $\overleftarrow{vw}$ are present in $G'$.)  Then $G'$ has maximum out-degree at most $4d$ and $\widehat{\mathcal{P}}_\ell(G')$ contains $\mathcal{P}$. Let $\gamma:\mathcal{P}\to V(G)$ be the result of applying \cref{advanced_cairns2} to $G'$ (and then restricting it to the paths in $\mathcal{P}$.)  For each problematic path $\Pi\in \mathcal{P}$, we call the vertex $\gamma(\Pi)$ \defin{problematic}.  We will recolour $\gamma(\Pi)$ in order to avoid the potential $\ell$-violation at $\Pi$.

  Let $P$ be the set of all problematic vertices.  In the second phase, we recolour every vertex in $P$ with a colour in a palette $\Phi_{q+1}$ of size $ck\log n + 1$ whose colours are all larger than all colours in $\Phi_0,\ldots,\Phi_q$.  Since each $\ell$-violation of $\varphi$ contains a vertex in $P$, this recolouring eliminates all the existing $\ell$-violations in $\varphi$.  More precisely, after this recolouring any remaining $\ell$-violation must have both endpoints whose colour is in $\Phi_{q+1}$.  So that this never happens, we will ensure that our recolouring is a proper colouring of $G^\ell[P]$.  By definition, this means that any path in $\mathcal{P}_\ell(G)$ with both endpoints in $P$ has endpoints of different colours and is therefore not an $\ell$-violation.

  By \cref{degeneracy_of_g_l}, $G^\ell$ is $O(\Delta^{\lfloor\ell /2\rfloor})$-degenerate.  Let $H$ be a directed acyclic graph obtained from $G^{\ell}$ in which each vertex has out-degree $O(\Delta^{\lfloor\ell /2\rfloor})$. We want to show that $G^\ell[P]$ has chromatic number at most $ck\log n+1$.
  % \pat{If this is not tight, then it's because we insist on a proper colouring of $G^\ell[P]$, but all we really need is that, any length-$\le\ell$ path $v_0,\ldots,v_r$ in $G$ with $v_0,v_r\in P$ has $\varphi(v_0)\neq \varphi(v_r)$ or $\varphi(w)>\varphi(v_0)$ for some $w\in\{v_1,\ldots,v_{r-1}\}$.}
  To do this, we will show that the maximum out-degree of $H[P]$ is at most $ck\log n$, with high probability.  In fact, we will show something stronger: that every vertex $p$ in $H$ has at most $ck\log n$ out-neighbours in $P$.

  From this point on, we fix some vertex $p$ of $H$ and study the random variable $|N_H^+(p)\cap P|$.  Instead of focusing on the set $P$ of problematic vertices, we focus instead on problematic paths.  For each problematic vertex $y$, some path $\Pi$ in $\gamma^{-1}(y)$ is problematic. Therefore,
  \begin{equation}
    |N^+_H(p)\cap P| \le \sum_{y\in N^+_H(p)} \left|\left\{\Pi\in\gamma^{-1}(y): \text{$\Pi$ is problematic}\right\}\right| =: X'_{p} \enspace .
    \label{path_upper_bound}
  \end{equation}
  Each path $\Pi\in\mathcal{P}$ is problematic with probability at most $1/k$ since $\Pi$ becomes problematic precisely when we choose the same colour for $w:=\tau(\Pi)$ that was already chosen for the other endpoint of $\Pi$.  Therefore,
  \begin{align*}
    \E\left(|N^+_H(p)\cap P|\right)
    & \le \frac{1}{k}\sum_{y\in N^+_H(p)}|\gamma^{-1}(y)| \\
    & \le \frac{1}{k}\cdot|N^+_H(p)|\cdot O(\Delta^{\lfloor\ell /2\rfloor-1}) \\
    & \le \frac{1}{k}\cdot O(\Delta^{2\lfloor\ell /2\rfloor-1}) \\
    & = O(\Delta^{\lfloor\ell /2\rfloor-1/2}\log^{-b} n) = O(k\log^{-2b} n) \enspace .
  \end{align*}
  This is a good sign. If the event set $\{\text{``$\Pi$ is problematic''}:\Pi\in\mathcal{P}\}$ were mutually independent then it would be a simple matter of applying a Chernoff bound.  Unfortunately this is not the case since, for each vertex $w$, all of the events in $\{\text{``$\Pi$ is problematic''}:\Pi\in\tau^{-1}(w)\}$ are all affected by the choice of colour for $w$.

  The remainder of the proof is more probability than graph theory.  We will use a tail estimate for sums of independent random variables due to Bernstein that is applicable when these random variables have sufficiently small variance.  The statement of Bernstein's Inequality and the calculations needed to apply it in this context are deferred to the next section. We use the rest of our time here to describe a random variable $X_p$ that stochastically dominates $X'_p\ge |N^+_H(p)\cap P|$ and is a sum of independent random variables.\footnote{A random variable $X$ \defin{stochastically dominates} a random variable $Y$ if $\Pr(X\ge x) \ge \Pr(Y\ge x)$ for all $x\in\R$.}

  For each vertex $w$ of $G$, let $\mathcal{P}_{p,w}:=\tau^{-1}(w)\cap (\bigcup_{y\in N^+_H(p)}\gamma^{-1}(y))$ and define
  \[
    X'_{p,w} :=\left|\left\{\Pi\in\mathcal{P}_{p,w}:\text{$\Pi$ is problematic}\right\}\right|
    \enspace .
  \]
  Since each problematic path $\Pi$ counted by the right-hand-side of \cref{path_upper_bound} becomes problematic when $w:=\tau(\Pi)$ is coloured, $X'_p=\sum_{w\in V(G)} X'_{p,w}$.

  Suppose $w\in L_j$ for some $j\in\{0,\ldots,q\}$.  For each $\alpha\in\Phi_j$, let $N_\alpha(p,w)$ be the number of paths in $\mathcal{P}_{p,w}$ that would have become problematic if we had set the colour of $w$ to $\alpha$. Then $\sum_{\alpha\in\Phi(w)} N_\alpha(p,w) \le \sum_{\alpha\in\Phi_j} N_\alpha(p,w) \le \sum_{\alpha\in\Phi_j} N_\alpha(w) = |\tau^{-1}(w)|$.

  Let $\alpha_1,\ldots,\alpha_k$ be the colours in $\Phi(w)$ ordered so that
  \[
    N_{\alpha_1}(p,w) \ge N_{\alpha_2}(p,w) \ge\cdots\ge N_{\alpha_k}(p,w) \enspace .
  \]
  Then, for each $i\in\{1,\ldots,k\}$, $iN_{\alpha_i}(p,w)\le\sum_{j=1}^i N_{\alpha_j}(p,w) \le |\tau^{-1}(w)|$, so
  \[
    N_{\alpha_i}(p,w)\le \frac{|\tau^{-1}(w)|}{i}=\frac{O(\Delta^{\lfloor\ell/2\rfloor})}{i} \enspace .
  \]
  Therefore, regardless of any random choices made before choosing the colour of $w$ and any random choices made after choosing the colour of $w$, the random variable $X'_{p,w}$ is dominated by a random variable $X_{p,w}:=\min\{O(\Delta^{\lfloor\ell/2\rfloor})/i,M\}$ where $i$ is chosen uniformly in $\{1,\ldots,k\}$.

  Therefore, $|N_H^+(p)\cap P|$ is dominated by a sum $X_p:=\sum_{w} X_{p,w}$ of mutually independent random variables.  In order to apply a concentration result to the random variable $X_p$, we need to  establish sufficiently strong properties on the individual terms $X_{p,w}$.  In the next section, we bound the expectation and variance of each $X_{p,w}$ so that we can apply Bernstein's Inequality to prove that $\Pr(X_p\ge ck\log n)\le n^{-\Omega(c)}$.  Then the union bound implies that $\Pr(\max\{X_p:p\in V(H)\} \ge ck\log n)\le n^{-\Omega(c)}$.  Thus, with high probability, the number of additional colours in $\Phi_{q+1}$ needed to recolour the vertices of $P$ in the second phase is $O(k\log n)$.  Since the total number of colours used in the first phase is $O(k\log n)$, we have
  \[
    \lrn(G) = O(k \log n) = O(\Delta^{\lfloor\ell/2\rfloor-1/2}\log^{5/4} n) \enspace . \qedhere
  \]
\end{proof}

With the proof of \cref{l_degenerate_and_degree} out of the way, we now show how it implies \cref{l_d_degenerate_upper_bound}.

\begin{proof}[Proof of \cref{l_d_degenerate_upper_bound}]
  Since $G$ is $d$-degenerate, it has at most $dn$ edges and the sum of its vertex degrees is at most $2dn$.  Let $\Delta:=n^{1/(\lfloor\ell/2\rfloor+1/2)}\log^{-x} n$ for some value $x$ to be discussed shortly.  Then the set $S:=\{v\in V(G):\deg_G(v)\ge \Delta\}$ has size at most $2dn/\Delta=2dn^{\texp}\log^x n$.  Since $G-S$ is $d$-degenerate and has maximum degree $\Delta$, \cref{l_degenerate_and_degree} implies that
  \begin{align*}
    \lrn(G-S) &
    \le c\Delta^{\lfloor\ell/2\rfloor-1/2}\log^{5/4} n \\
    % cd^{3/2}\sqrt{\Delta}\log^{1/2} n
    % = c\cdot \sqrt{\frac{n^{2/3}}{d^{1/3}\log^{1/2} n}}\cdot \log^{3/4} n
    & = cn^{\frac{\lfloor\ell/2\rfloor-1/2}{\lfloor\ell/2\rfloor+1/2}}\log^{5/4-x(\lfloor\ell/2\rfloor-1/2)} n \\
    & = cn^{\dexp}\log^{5/4-x(\lfloor\ell/2\rfloor-1/2)} n \\
    & \le cn^{\dexp}\log^{5/4-x/2} n \\
    % & = O\left(n^{\dexp}\right)
  \end{align*}
  where the last inequality follows from the fact that $\lfloor\ell/2\rfloor-1/2\ge 1/2$ for all $\ell\ge 2$.  Taking $x:=5/6$, we get $|S|=O(n^{\texp}\log^{5/6} n)=O(n^{\texp}\log n)$ and  $\lrn(G-S)= O(n^{\texp}\log^{5/4-5/12} n)=O(n^{\texp}\log^{5/6} n)$.  We can extend $\varphi$ to a colouring of $G$ by assigning each vertex in $S$ a distinct colour that is larger than every colour used in the colouring of $G-S$.  Thus, $\trn(G)\le |S|+\lrn(G-S) \le O(n^{\texp}\log^{5/6} n)$, which establishes \cref{l_d_degenerate_upper_bound}.  (Note that this argument actually proves a slightly better bound than \cref{l_d_degenerate_upper_bound}.  For $\ell\ge 4$, further improvements to the logarithmic factor in \cref{l_d_degenerate_upper_bound} can be obtained using the ``Furthermore'' clause of \cref{l_degenerate_and_degree}.)
\end{proof}

%



% \section{A Matching Lower Bound?}
%
% % \begin{thm}
% %   There exists a constant $c$ such that, for every even integer $\ell$ and every integer $n\ge 1$, there exists a $2$-degenerate graph $G$ with $O(n\log n)$ vertices having $\lrn(G)\ge n^{1-2/\ell}$.
% % \end{thm}
% %
% % \begin{proof}
% %   The case $\ell=2$ is trivial.  For $\ell\ge 4$, let $G_0$ be a random graph having $n_0$ vertices in which each vertex chooses $n_0^{2/(\ell-2)}\log n_0$ neighbours uniformly at random.  Then $G_0$ has $O(n_0^{\ell/(\ell-2)}\log n_0)$ edges and
% %   \begin{compactenum}[(a)]
% %     \item with high probability the treewidth of $G_0$ is $\Omega(n_0)$;
% %     \item with high probability the diameter of $G_0$ is at most $(\ell-2)/2$.
% %   \end{compactenum}
% %   Let $G$ be the graph obtained by subdividing each edge of $G_0$.  Then $G_0$
% %   \begin{compactenum}
% %     \item the treewidth of $G_0$
% %   \end{compactenum}
% % \end{proof}
%
%
%
%
% Here's a non-trivial lower-bound, for $\ell=4$.  Start with a complete bipartite graph $K_{k,2k}$ with parts $A$ and $B$ and subdivide each edge once.  Call this graph $G$.  Then $G$ has $n=2k+4k^2$ vertices.  I claim that any $4$-vertex-ranking of this graph requires $k=\Omega(\sqrt{n})$ colours.  If we use less than $k$ colours on $A$ then two vertices $v,w\in A$ receive the same colour $c$.  There are $2k$ paths $P_1,\ldots,P_{2k}$ of length $4$ from $v$ to $w$ in $G$ that are pairwise vertex-disjoint except for their common endpoints.  Since $v$ and $w$ have the same colour, the maximum colour in each $P_i$ must be larger than $c$.  Classify each $P_i$ as a $v$-path or a $w$-path if the maximum colour that appears on $P_i$ is within distance $2$ of $v$ or $w$, respectively (a path can be both if its middle vertex has the highest colour).  Without loss of generality, at least $k$ of the paths are $v$-paths.  But then the maximum colours on each of these paths must be distinct since, otherwise, there is a path through $v$ with two endpoints whose colours are the same and larger than any other colour on the path.
%
% This gives a lower bound of $\Omega(n^{1/2})$ for $\ell=4$ and our upper bound in this case is $O(n^{3/5}\log n)$.  I don't see how to get any further with constructions like this. The problem is that in order to prove a lower bound of $k$, this construction requires a set $A$ of $k$ vertices such that, for any two vertices in $A$, there are $k$ interior-disjoint paths between them.  This requires that each vertex in $A$ has degree at least $k$, for a total degree of $k^2$.  But if the graph is $d$-degenerate, then it has total degree $dn \ge k^2$, so $k\le \sqrt{dn}$.  In other words, we can never prove a bound greater than $\sqrt{dn}$ this way, no matter how big we make $\ell$.  Another way to think of it is that any construction we make should not have vertices of degree greater than $2dn/k$, since there are at most $k$ such vertices so we could remove them beforehand and give them all distinct large colours later.  If we're shooting for the optimal bound $k=n^{3/5}$ for $\ell=4$ or $\ell=5$ then we shouldn't have any vertices of degree greater than $n^{2/5}=k^{2/3}$.
%
% Here is something even more non-trivial, but still not optimal.
%
% \begin{thm}
%   There exists a $2$-degenerate $n$-vertex graph $G$ with $\rn{4}(G)\ge \Omega(n^{5/9})$.
% \end{thm}
%
% \begin{proof}[Proof Sketch]
%   Let $n_0:=n^{5/9}$, let $\delta:=n^{4/9}$, and let $p:=\delta/n_0$. Construct a directed graph $G_0$ of size $2n_0$ as follows:  For each pair $\{v,w\}\in \binom{V(G_0)}{2}$ add the directed edge $\overrightarrow{vw}$ with probability $p$ and add the directed edge $\overleftarrow{vw}$ with probability $p$, with all choices independent.  With high probability, the number of edges of $G_0$ is $O(\delta n_0)=O(n)$.  The graph $G$ we consider is obtained by subdividing each edges of $G_0$ and ignoring directions. Then $G$ is $2$-degenerate and, with high probability, $|V(G)|=O(n)$. Note that any vertex-colouring $\varphi:V(G)\to\{1,\ldots,k\}$ of $G$ corresponds to a total colouring $\varphi:V(G_0)\cup E(G_0)\to\{1,\ldots,k\}$.  We will therefore focus on total colourings of $G_0$ and more-or-less forget about $G$.
%
%   Fix a colouring $\varphi:V(G_0)\to\{1,\ldots,k\}$ that is independent of the random choices made when constructing $G_0$.  (Later we will use the union bound on all $k^{n_0}$ such colourings.)  By doubling the number of colours and eliminating infrequent colours, we can assume that all colour classes have size in the range $[n_0/2k,n_0/k]$; in the following sketch, we'll assume that they all have size $n_0/k$.  Let $A:=\{v\in V(G):\varphi(v)\le k/2\}$ and let $B:=\{w\in V(G):\varphi(w)> k/2\}$.  By our assumptions, $A$ and $B$ each have size $n_0$ and from now on we will focus entirely on the subgraph of $G_0$ containing only the edges with one endpoint in each of $A$ and $B$.
%
%   For each $\alpha\in\{k/2+1,\ldots,k\}$ and each $v\in A$ we say that $v$ is \defin{$\alpha$-bad} if there are two edges $\overrightarrow{uv}$ and $\overrightarrow{vw}$ with $\varphi(u)=\varphi(w)=\alpha$.  (In our terminology, $uvw$ is a $2$-violation of $\varphi$.)  It is easy to check that
%   \[
%     \Pr(\text{$v$ is $\alpha$-bad}) \ge \Omega((n_0/k)^2p^2) = \Omega(\delta^2/k^2)
%   \]
%   (The number of edges from $v$ into colour class $\alpha$ is a binomial$(n_0/k,p)$ random variable and $v$ is $\alpha$-bad if this variable takes on the value $2$.)  For each $\alpha$-bad vertex $v\in A$, let $B_{v,\alpha}$ be some pair of edges $\overrightarrow{vu}$ and $\overrightarrow{vw}$ with $\varphi(u)=\varphi(w)=\alpha$.  Now note that the size of the set $Z:=\{(v,\alpha)\in A\times\{k/2+1,\ldots,k\}: \text{$v$ is $\alpha$-bad}\}$ is dominated by a binomial$(n_0 k/2,\Omega(\delta^2/k^2))$ random variable and with high probability $|Z|=\Omega(n_0\delta^2/k)$.
%
%   If the colouring $\varphi$ can be extended to a $4$-vertex-ranking of $G$ then, for each $(v,\alpha)\in Z$, at least one of the two edges $vw$ or $vu$ in $B_{\alpha,v}$ must receive a colour larger than $\alpha$ since, otherwise $uvw$ corresponds to a $4$-violation in $G$. That is, $\varphi(vw)\ge \alpha=\varphi(w)> k/2\ge \varphi(v)$.  For each $(v,\alpha)\in Z$, choose one edge $e_{v,\alpha}\in B_{v,\alpha}$ and let $X:=\{e_{v,\alpha}:(v,\alpha)\in Z\}$ be the set of all edges chosen this way. Note that there only $2^{|Z|}=2^{|X|}$ choices for $X$ and that the definition of $X$ depends only on the choice of directed edges from $A$ into $B$.  It does not depend on the choice of edges from $B$ into $A$. To extend $\varphi$ to a $4$-vertex-ranking of $G$ there must be some way to choose such an $X$ and then assign, for each $(v,\alpha)\in Z$ a colour $\varphi(e_{v,\alpha})>\alpha$.
%
%   Now, fix a colouring $\varphi:X\to\{1,\ldots,k\}$.  Note that there is no hope of extending this colouring if two edges of $X$ incident to the same vertex $v\in V(G)$ receive the same colour, so we may assume that $\varphi$ is a proper edge-colouring of $G_0[X]$.  We say that a pair of distinct edges $\{\overrightarrow{vw},\overrightarrow{xy}\}\in X$ is \defin{dangerous} if $\varphi(vw)=\varphi(xy)$.  If the edge  $\overleftarrow{vy}$ or the edge $\overleftarrow{xw}$ is present in $G_0$ then we say that these edges are \defin{dangerous} edges of $G_0$.  Note that, if $\overleftarrow{vy}$ is dangerous and we want to extend $\varphi$ to a $4$-vertex-ranking of $G$ then we must have $\varphi(vy)>\varphi(vw)\ge\varphi(w)>\varphi(v)$ since, otherwise, some path $xyvw$ has $\varphi(xy)=\varphi(vw)\ge \varphi(yv)\ge\varphi(y)>\varphi(v)$, which corresponds to a $4$-violation in $G$.  In particular, all of the dangerous edges incident to a particular vertex $z\in A\cup B$ must receive distinct colours, otherwise $G$ contains a $2$-violation at $z$.  All that remains is to show that the set $D$ of dangerous edges is too large for this to happen.
%
%   To do this, we will argue that there is a one-to-one function $f:\binom{X}{2}\to A\times B$ such that, if $f(\{e_1,e_2\})=(v,w)$ then $(v,y)$ has the potential to be a dangerous edge $\overleftarrow{vy}$ of $G_0$.  In this way, the size of $D$ is lower-bounded by a binomial$(m,p)$ random variable where $m$ is the number of dangerous pairs in $X$. We are able to do this because $\varphi$ is a proper edge-colouring of $G[X]$, so the number of $v\in A$ such that $v$ is $\alpha$-bad is equal to the number of $w\in B$ such that $w$ is incident to an edge of $X$ whose colour is $\alpha$.  adjacent to $\alpha$-bad vertex $v\in A$.  all the edges of $G$ incident to a vertex $v$ [Finish this with an augmenting path argument.]
%
%   The colouring of $X$ partitions the edges in $X$ into at most $k$ colour classes $X_1,\ldots,X_k$.  Each colour class $X_k$ determines $2\binom{|X_k|}{2}$ pairs of vertices that potentially form dangerous edges. Therefore, the number of dangerous pairs of edges in $X$ is at least
%   \[
%     \sum_{i=1}^k 2\binom{|X_k|}{2} \ge k\cdot\Omega((|Z|/k)^2) =   k\cdot\Omega\left(\left(\frac{n_0\delta^2}{k^2}\right)^2\right)
%     = \Omega\left(\frac{n_0^2\delta^4}{k^3}\right) \enspace .
%   \]
%   There $|D|$ dominates a binomial$(n_0^2\delta^4/k^3,p)$ random variable, so with high probability,
%   \[
%      |D| \ge p\cdot \Omega\left(\frac{n_0^2\delta^4}{k^3}\right)
%      = \Omega\left(\frac{n_0\delta^5}{k^3}\right) \enspace .
%   \]
%   % Unfortunately, $|D|$ is not a binomial random variable.
%   But this means that some vertex is adjacent to $|D|/n_0=\Omega(\delta^5/k^3)$ edges in $D$, and these edges must all receive distinct colours.  Therefore,
%   \[
%     k\ge \Omega\left(\frac{\delta^5}{k^3}\right)
%     \quad\Leftrightarrow\quad
%     k \ge \Omega(\delta^{5/4})=\Omega(n^{5/9}) \enspace .
%   \]
%   Thus, if $\varphi:V(G)\to\{1,\ldots,k\}$ is a $4$-vertex-ranking of $G$ then there exists a certificate that consists of three parts:
%   \begin{compactenum}[(a)]
%     \item a colouring $\varphi:V(G_0)\to\{1,\ldots,k\}$. The number of these is
%     \[ k^{n_0} = \exp(n_0\log k) = \exp(n^{5/9}\log n) \enspace . \]
%     \item a choice of the set $X$. The number of choices for $X$ is
%     \[  2^{\Theta(n_0\delta^2/k)}=\exp(\Theta(\delta^2))=\exp(\Theta(n^{8/9})) \enspace . \]
%     \item a colouring $\varphi:X\to\{1,\ldots,k\}$. The number of such colourings is
%     \[  k^{\Theta(n_0\delta^2/k)}=\exp(\Theta(\delta^2\log k))=\exp(\Theta(n^{8/9}\log n)) \]
%   \end{compactenum}
%   Our random graph fails for part (a) of the certificate if the set $Z$ is to small.  Since $|Z|$ is a binomial random variable, this happens with probability at most $\exp(-\E(|Z|))$.  Therefore, the vast majority of choices for $G_0$ will pass for any choice of part (a) of the certificate, where ``pass'' means that $|Z|\ge cn_0\delta^2/k$ since, for $k \le n_0/2$,
%   \[
%      \exp\left(-\E(Z)\right)\le \exp(-\Omega(\delta^2 n_0/k)) =  \exp(-\Omega(\delta^2)) = \exp(-\Omega(n^{8/9}))
%   \]
%   Our random graph fails for part (b) and (c) of the certificate if the set $D$ of dangerous edges is too small.  Again, since $|D|$ is a binomial random variable this happens with probability
%   \[
%       \exp(-\E(|D|))=\exp\left(-\frac{n_0\delta^5}{k^3}\right)
%       = \exp\left(-\frac{n^{25/9}}{k^3}\right) \le \exp\left(-n^{10/9}\right)
%   \]
%   Therefore, the vast majority of graphs $G_0$ pass for all every part (a) certificate and the vast majority of graphs $G_0$ pass for every part (b) and (c) certificate.  Therefore, the vast majority of graphs $G_0$ pass for all certificates.
% \end{proof}
%



\section{Bounding the Tail of \boldmath $X_p$}
\label{probability}

We make use of the following inequality of Bernstein \cite[Corollary~2.11]{boucheron.lugosi.ea:concentration}:

\begin{thm}\label{bernstein_theorem}
  Let $M$ be a positive number, let $X_1,\ldots,X_r$ be independent random variables such that $0\le X_i\le M$ for each $i\in\{1,\ldots,r\}$, and let $X:=\sum_{i=1}^r X_i$. Then
  \begin{equation}
    \Pr\left(X \ge \E(X)+ t\right)
      \le \exp\left(\frac{\tfrac{1}{2}t^2}{\sum_{i=1}^r \E((X_i-\E(X_i))^2)+\tfrac{1}{3}Mt}\right) \enspace . \label{bernstein}
  \end{equation}
\end{thm}
We will apply \cref{bernstein_theorem} to a random variable $X:=\sum_{i=1}^r X_i$ in which each $X_i$ has the following distribution (for some $0\le x_i\le kM$):
\[
  X_i = \begin{cases}
          M & \text{with probability $(1/k)\lfloor x_i/M\rfloor$} \\
          x_i/j & \text{with probability $1/k$ for each $j\in\{\lfloor x_i/M\rfloor+1,\ldots,k\}$}
        \end{cases}
\]
This is the distribution we get when we choose a uniform $j\in\{1,\ldots,k\}$ and set $X_i:=\min\{M,x_i/j)$.  To see how this applies in the proof of \cref{l_degenerate_and_degree}, let $r:=n$, let $\{w_1,\ldots,w_n\}:=V(G)$ and, for each $i\in\{1,\ldots,n\}$, let $x_i:=|\tau^{-1}(w_i)\cap\left(\bigcup_{y\in N^+_H(p)}\gamma^{-1}(y)\right)|$.  In our setting $k=\Delta^{\lfloor\ell/2\rfloor-1/2}\log^b n$ for some $b\ge 0$,  $M=a\Delta^{\lfloor\ell/2\rfloor}/k$ for some constant $a$, and $t=ck\log n$ for some (sufficiently large) constant $c$.  The rest of this appendix is devoted to bounding the various quantities that appear in \cref{bernstein} so that we can show that the right-hand side of \cref{bernstein} is $n^{-\Omega(c)}$.

Both the maximum value and the sum of $x_1,\ldots,x_n$ are important for us. For the maximum, we have $x_i\le|\tau^{-1}(w_i)|= O(\Delta^{\lfloor\ell/2\rfloor})$ for all $i\in\{1,\ldots,n\}$.  We have already bounded the sum when computing the expectation of $|N_H^+(p)\cap P|$ as follows:
\[
  \sum_{i=1}^n x_i = \sum_{y\in N^+_H(p)} |\gamma^{-1}(y)|
  \le |N^+_H(p)|\cdot O(\Delta^{\lfloor\ell/2\rfloor-1})
  \le O(\Delta^{2\lfloor\ell/2\rfloor-1}) \enspace .
\]

For each $i\in\{1,\ldots,n\}$, we have
\begin{align*}
  \E(X_i)
  & =\Pr(X_i=M)\cdot M + \sum_{j=\lfloor x_i/M\rfloor+1}^k \Pr(X_i=x_i/j)\cdot\frac{x_i}{j} \\
  & \le\frac{x_i}{kM}\cdot M + \frac{1}{k}\cdot\sum_{j=\lfloor x_i/M\rfloor+1}^k \frac{x_i}{j} \\
  & \le\frac{x_i}{k} + \frac{1}{k}\cdot\sum_{j=1}^k \frac{x_i}{j} \\
  & \le \frac{x_i(2+\ln k)}{k} & \text{(since $\sum_{j=1}^k 1/k\le 1+\ln k$)} \\
  & = O\left(\frac{x_i\log k}{k}\right) \\
  & = O\left(\frac{x_i\log n}{k}\right)
  \enspace ,
\end{align*}
where the last line comes from the fact that $d,\Delta \le n$ and $\ell\in O(1)$.
Therefore,
\begin{align}
  (\E(X_i))^2
  & = O\left(\left(\frac{x_i\log n}{k}\right)^2\right) \nonumber \\
  & = O\left(\frac{x_i^2\log^2 n}{k^2}\right) \nonumber \\
  & = O\left(\frac{x_i^2\log^2 n}{\Delta^{2\lfloor\ell/2\rfloor-1}\log^{2b} n}\right) \nonumber \\
  & = O\left(\frac{x_i\log^{2-2b} n}{\Delta^{\lfloor\ell/2\rfloor-1}}\right)
   & \text{(since $x_i=O(\Delta^{\lfloor\ell/2\rfloor})$)} \label{square}
\end{align}
Now,
\begin{equation}
  \E((X_i-\E(X_i))^2)  = \frac{1}{k}\left\lfloor\frac{x_i}{M}\right\rfloor\cdot(M-\E(X_i))^2 + \sum_{j=\lfloor x_i/M\rfloor+1}^k \frac{1}{k}\left(\frac{x_i}{j}-\E(X_i)\right)^2 \label{variance}
\end{equation}
We bound the first term on the right hand side of \cref{variance} as follows:
\begin{align*}
  \frac{1}{k}\left\lfloor\frac{x_i}{M}\right\rfloor\cdot(M-\E(X_i))^2
  & \le \frac{x_i}{kM}\left(M^2 + (\E(X_i))^2\right) \\
  & = \frac{M x_i}{k} + \frac{x_i}{kM}\cdot (\E(X_i))^2 \\
  & \le \frac{M x_i}{k} + (\E(X_i))^2
  & \text{(since $x_i=|\tau^{-1}(w_i)|\le Mk$)}\\
  & = O\left(\frac{\Delta^{\lfloor\ell/2\rfloor}x_i}{k^2}\right) + (\E(X_i))^2
  & \text{(by the definition of $M$)}\\
  & = O\left(\frac{x_i\log^{-2b} n}{\Delta^{\lfloor\ell/2\rfloor-1}}\right) + (\E(X_i))^2
  & \text{(by the definition of $k$)}\\
  & = O\left(\frac{x_i\log^{2-2b} n}{\Delta^{\lfloor\ell/2\rfloor-1}}\right)
  & \text{(by \cref{square}, since $2-2b\ge -2b$)} \enspace .
\end{align*}
We bound the remaining terms in \cref{variance} as follows:
\begin{align*}
 \sum_{j=\lfloor x_i/M\rfloor+1}^k \frac{1}{k}\left(\frac{x_i}{j}-\E(X_i)\right)^2
  & \le \sum_{j=\lfloor x_i/M\rfloor+1}^k \frac{1}{k}\left(\frac{x_i^2}{j^2}+(\E(X_i))^2\right) \\
  & \le (\E(X_i))^2+\frac{1}{k}\sum_{j=\lfloor x_i/M\rfloor+1}^k \frac{x_i^2}{j^2} \\
  & \le (\E(X_i))^2 +  \frac{1}{k}\cdot\frac{M\pi^2x_i^2}{6 x_i}
  & \text{(since $\sum_{j=z}^{\infty} \tfrac{1}{j^2} \le \tfrac{\pi^2}{6z}$ for $z\ge 1$)} \\
  & = (\E(X_i))^2 + O\left(\frac{Mx_i}{k}\right) \\
  % & = (\E(X_i))^2 + O\left(\frac{x_i\log^{-2b} n}{\Delta^{\lfloor\ell/2\rfloor-1}}\right) & \text{(as above)} \\
  & = O\left(\frac{x_i\log^{2-2b} n}{\Delta^{\lfloor\ell/2\rfloor-1}}\right)
  & \text{(as above)} \enspace .
\end{align*}
Therefore
\[
  \E((X_i-\E(X_i))^2) = O\left(\frac{x_i\log^{2-2b} n}{\Delta^{\lfloor\ell/2\rfloor-1}}\right)
\]
for all $i\in\{1,\ldots,n\}$.  Recall that $\sum_{i=1}^n x_i = O(\Delta^{2\lfloor\ell/2\rfloor-1})$, so
\[
  \E(X)
  = \sum_{i=1}^n \E(X_i)
  = \sum_{i=1}^n O\left(\frac{x_i\log n}{k}\right)
  = O\left(\frac{\Delta^{2\lfloor\ell/2\rfloor-1}\log n}{k}\right)
  = O\left(\Delta^{\lfloor\ell/2\rfloor-1/2}\log^{1-b} n\right) \enspace .
\]
and
\[
  \sum_{i=1}^r\E((X_i-\E(X_i))^2)
  = \sum_{i=1}^r O\left(\frac{x_i\log^{2-2b} n}{\Delta^{\lfloor\ell/2\rfloor-1}}\right)
  = O\left(\frac{\Delta^{2\lfloor\ell/2\rfloor-1}\log^{2-2b} n}{\Delta^{\lfloor\ell/2\rfloor-1}}\right)
  = O\left(\Delta^{\lfloor\ell/2\rfloor}\log^{2-2b} n\right)
  \enspace .
\]
Then \cref{bernstein} gives:
\begin{align*}
  \Pr(X\ge \E(X)+t)
  & = \Pr(X\ge \E(X)+ck\log n) \\
  & \le \exp \left(-\frac{\tfrac{1}{2}(ck\log n)^2}{O\left(\Delta^{\lfloor\ell/2\rfloor}\log^{2-2b} n\right) + \tfrac{1}{3}Mck\log n}\right) \\
  & = \exp \left(-\frac{\tfrac{1}{2}(ck\log n)^2}{O\left(\Delta^{\lfloor\ell/2\rfloor}\log^{2-2b} n +c\Delta^{\lfloor\ell/2\rfloor}\log n)\right)}\right)
    & \text{(since $Mk=O(\Delta^{\lfloor\ell/2\rfloor})$)} \\
  & = \exp \left(-\frac{\tfrac{1}{2}(ck\log n)^2}{O\left(c\Delta^{\lfloor\ell/2\rfloor}\log^{2-2b} n\right)}\right) & (\text{for $c\ge 1$ and $b\le 1/2$}) \\
  & = \exp \left(-\frac{\tfrac{1}{2}c\Delta^{2\lfloor\ell/2\rfloor-1}\log^{2+2b} n}{O\left(\Delta^{\lfloor\ell/2\rfloor}\log^{2-2b} n\right)}\right) & \text{(by definition of $k$)}\\
  & = \exp\left(-\Omega(c\Delta^{\lfloor\ell/2\rfloor-1}\log^{4b} n)\right) \\
  % & = \exp\left(-\Omega(c\log^{4b} n)\right) & \text{since $\ell\ge 2$} \\
  & = n^{-\Omega(c)} \enspace ,
\end{align*}
provided that $b\ge 1/4$ or $\Delta^{\lfloor\ell/2\rfloor-1}\ge\log n$.
For any fixed $c$, taking $b=1/4$ gives
\begin{align*}
  \E(X)+t
    & = O(\Delta^{\lfloor\ell/2\rfloor-1/2}\log^{1-b} n + k\log n) \\
    & = O(\Delta^{\lfloor\ell/2\rfloor-1/2}(\log^{1-b} n + \log^{1+b} n)) \\
    & = O(\Delta^{\lfloor\ell/2\rfloor-1/2}\log^{5/4} n)
\end{align*}
Thus, $\Pr(X\ge c\Delta^{\lfloor\ell/2\rfloor-1/2}\log^{5/4} n) \le n^{-\Omega(c)}$, which completes the first part of the proof of \cref{l_degenerate_and_degree} for the cases $\ell=2$ and $\ell=3$.  To complete the ``furthermore'' clause of the proof we take $b=0$ and deduce that $\E(X)+t=O(\Delta^{\lfloor\ell/2\rfloor-1/2}\log n)$.


\section*{Acknowledgement}

Some of this research began during the \emph{Third Workshop on Geometry and Graphs (WoGaG~2015)}, March 8–13, 2015, and resumed during the \emph{Second Adriatic Workshop on Graphs and Probability (AWGP~2023)}, Jun 24–Jul 1, 2023. The authors are grateful to the organizers of both workshops for providing a stimulating and productive work environment.


\bibliographystyle{plainnat}
\bibliography{us2}


% We finish by noting that, for $\ell\ge 4$ and $\Delta\ge\log n$, we can drop the condition $b\ge 4$ and the calculation still works because the $\Delta^{\lfloor\ell/2\rfloor-1}$ factor makes up for the $\log^{4b} n$ factor.  In particular, the proof of \cref{l_degenerate} has $\Delta

\end{document}
